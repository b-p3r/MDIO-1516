\chapter{Parte 2}
\label{cap:p2}

\section{Análise Problema}

O problema do Trabalho 1 tratava da descoberta do tempo em que cada
atividade é iniciada, sabendo que todas as atividades são realizadas.



\section{Modelo Primal --- Trabalho 1}

\subsection{Parâmetros}

Os parâmetros do problema são a duração de cada atividade e as suas
precedências.

\subsection{Variáveis de decisão}

As variáveis de decisão correspondem ao tempo em que cada atividade é iniciada.
Assim, a cada atividade está associada uma variável de decisão. Relativamente ao
nome, a opção tomada foi a de considerar $T_{i}$ como o tempo de início da
atividade $i$ (em unidades de tempo arbitrárias), em que $i$ corresponde ao
número da atividade. Uma vez que apenas se pretende conhecer os tempos de início
de cada atividade, estas são as únicas variáveis deste modelo. 

\subsection{Função Objetivo}

Neste modelo, quer-se minimizar o tempo de execução total do projeto. Isso
corresponde a dizer que queremos que a atividade final seja iniciada o mais cedo
possível. A atividade final é na verdade ``fictícia'', pois não corresponde
a uma atividade que tenha de ser efetivamente realizada. De igual modo,
atividade inicial é  ``fictícia''. No entanto para efeitos de modelação, é útil
considerá-las, para efeitos de modelação do modelo dual. Assume-se que
a atividade final é realizada após todas as outras da rede terem terminado e que
tem duração de 0 unidades de tempo. A atividade inicial tem, de igual modo,
duração de 0 unidades de tempo, sendo pertinente usá-la para a transfonação no
modelo dual.Nestas condições, o tempo inicial da atividade final indica
a duração do projeto.


Uma vez que a variável $T_{fim}$ indica a duração do projeto, a função
objetivo fica simplesmente:

\begin{displaymath} 
	\min~z = T_{fim}-T_{ini} 
\end{displaymath}

A variável $T_{ini}$ está aqui para efeitos de futura modelação do dual.

\subsection{Restrições}

Com as restrições pretende-se indicar o espaço de possíveis soluções. Sabe-se
que uma atividade não pode começar sem que as que lhe precedem tenham terminado.
Qualquer solução que obedeça a este princípio é uma solução admissível para
o problema. Para escrever as restrições é por isso necessário saber quando uma
atividade termina. Ora, sabendo que as variáveis de decisão usadas indicam
o tempo em que cada atividade se inicia e que se tem a duração das mesmas como
parâmetro do modelo, pode-se dizer que o tempo final de uma atividade
corresponde a somar o seu tempo de início com a sua duração. Ou seja:

\begin{displaymath} Tf_{i} = T_{i} + D_{i} \end{displaymath}

Onde:

\begin{itemize} \item[$Tf_{i}$] Tempo em que a atividade $i$ termina
		\item[$T_{i}$] Tempo em que a atividade i começa (variável de decisão)
		\item[$D_{i}$] Duração da atividade $i$ \end{itemize}

Dizer que uma atividade não pode começar sem que as que lhe precedem tenham
terminado é o mesmo que dizer que o tempo inicial da atividade tem que ser maior
que o tempo final de todas as atividades que lhe precedem. Assumindo que se tem
uma atividade $j$ que precede uma atividade $i$, pode-se escrever que:

\begin{displaymath} T_{i} \geq T_{j} + D_{j} \end{displaymath}

O modelo terá por isso uma restrição deste tipo por cada nodo e por cada
atividade precedente ao nodo. Ou seja, um nó que tenha apenas 1 precedência,
apenas originará uma restrição, enquanto que se o nodo tiver por exemplo
3 precedências, dará origem a 3 restrições --- uma restrição para cada precedência
do nodo. 
As restrições completas:

\begin{itemize}

\item{Nodo Inicial
	\begin{align*}
		T_ {ini} \ge 0 + 0 &         &         &         &         &         &
	\end{align*}}

\item Nodo 0
	\begin{align*}
		T_0 \ge T_{ini} + 0 &         &         &         &         &         &
	\end{align*}


\item Nodo 1
	\begin{align*}
T_1 \ge T_0 + 4 &         &         &         &         &         &
	\end{align*}


\item Nodo 3
\begin{align*}
T_3 \ge T_1 + 6 &         &         &         &         &         &\\
T_3 \ge T_5 + 4 &         &         &         &         &         &\\
T_3 \ge T_4 + 9 &         &         &         &         &         &
	\end{align*}



\item Nodo 4
\begin{align*}
T_4 \ge T_0 + 4 &         &         &         &         &         &\\
T_4 \ge T_7 + 6 &         &         &         &         &         &
	\end{align*}


\item Nodo 5
\begin{align*}
        T_5 \ge T_4 + 9  &         &         &         &         &         &\\
        T_5 \ge T_7 + 6  &         &         &         &         &         &\\
        T_5 \ge T_{10} + 8  &         &         &         &         &         &
	\end{align*}


\item Nodo 6
	\begin{align*}
	T_6 \ge T_{ini} + 0 &         &         &         &         &         &
	\end{align*}

\item Nodo 7
\begin{align*}
       T_7 \ge T_6 + 5  &         &         &         &         &         &\\
	\end{align*}


\item Nodo 9
\begin{align*}
T_9 \ge T_7 + 6   &         &         &         &         &         &\\
T_9 \ge T_{11} + 7 &         &         &         &         &         &\\
T_9 \ge T_{10} + 8 &         &         &         &         &         &
	\end{align*}


\item Nodo 10
\begin{align*}
		T_{10} \ge T_6 + 5 &         &         &         &         &         &
	\end{align*}

\item Nodo 11
\begin{align*}
		T_{11} \ge T_{10} + 8 &         &         &         &         &         &	
\end{align*}



\item Nodo final
\begin{align*}
		T_{fim} \ge T_3 + 2 &         &         &         &         &         &\\
T_{fim} \ge T_5 + 4 &         &         &         &         &         &\\
T_{fim} \ge T_9 + 2 &         &         &         &         &         &
	\end{align*}


\end{itemize}


Visto que os tempos não podem ser negativos, neste modelo tem-se ainda
restrições de não-negatividade:

\begin{displaymath} T_{i} \geq 0, \forall i_{\in\{ini, 0, 1, 3,
	4,5,6,7,9,10,11,fim\}} \end{displaymath}


\section{Construção do modelo dual do Trabalho 1}


\subsection{Variáveis de decisão}

Dado o problema primal anterior, é  necessário associar variáveis de decisão
duais ao modelo $Y_i$, para cada restrição i do problema anterior descrito. Por
exemplo, uma restrição no modelo primal é $T_{ini} \ge 0$. Como essa é a primeira
linha, então a variável dual associada a essa restrição é $Y1$.

Cada variável de decisão na função objetivo tem um custo associado. A variável
$T_{fim}$ tem custo 1 e a variável $T_{ini}$ tem custo -1. Todas as outras variáveis
$T_i \forall i_{\in\{0, 1, 3, 4,5,6,7,9,10,11\}}$ têm o custo 0. No modelo dual,
estes valores corresponderão ao coeficiente de valor constante em cada
inequação.

\subsection{Restrições}
Para cada variável de decisão do problema primal $ T_i \forall i_{\in\{ini, 0,
1, 3,4,5,6,7,9,10,11,fim\}}$ procuraram-se ocorrências destas nas linhas de cada
restrição, sendo depois associada a variável $Y_i$ dessa linha, multiplicada
pelo custo de $T_i$ no modelo primal que ocorre nessa linha. Por exemplo,
a variável $T_{ini}$ ocorre na linhas de $Y_1$, $Y_2$ $Y_{12}$, sendo o seu
custo 1, -1 e -1, respetivamente. Como já foi dito, o custo da variável de de
decisão na função objetivo passa a ser o valor do coeficiente constante da
inequação da restrição, associado a $T_i$. O problema primal é de minimização
e está na forma canónica (restrições de $\ge$). Então o dual é de maximização
e todas as restrições serão de $\le$.Para o exemplo dado, a nova restrição do
modelo dual passa a ser: $$Y_1 - Y_2 - Y_{12} \le -1$$


\subsection{Função objetivo}
Para a função objetivo do modelo dual, cada valor dos coeficientes constantes
das restrições do modelo primal será o custo associado à variável $Y_i$ nessa
linha da restrição correspondente. Por exemplo, para $X_{ini} \ge 0$ cuja
variável dual associada é $Y_1$, na função objetivo do modelo dual estará
$0 \times Y_i$.

De igual modo, como referido anteriormente, o modelo primal é de maximização,
logo o seu dual é de minimização.

\newpage
\subsection{Construção do modelo dual --- concretização}
Neste passo, colocaram-se em evidencia os valores constantes das restrições do
modelo primal, correspondentes às durações das atividades, assim como as
variáveis duais $Yi$ para cala linha. Assim, temos que:


\begin{verbatim}

 - Nodo Inicial
Tini  >=  0       ...... Y1

 - Nodo 0
T0  -  Tini >= 0  ...... Y2

 - Nodo 1
T1  -  T0  >= 4   ...... Y3

 - Nodo 3
T3  -  T1  >= 6   ...... Y4
T3  -  T5  >= 4   ...... Y5
T3  -  T4  >= 9   ...... Y6

 - Nodo 4
T4  -  T0  >= 4   ...... Y7
T4  -  T7  >= 6   ...... Y8

 - Nodo 5
T5  -  T4  >= 9   ...... Y9
T5  -  T7  >= 6   ...... Y10
T5  -  T10  >= 8  ...... Y11

 - Nodo 6
T6  -  Tini  >= 0 ...... Y12

 - Nodo 7
T7  -  T6  >= 5   ...... Y13

 - Nodo 9
T9  -  T7  >= 6   ...... Y14
T9  -  T11  >= 7  ...... Y15
T9  -  T10  >= 8  ...... Y16

 - Nodo 10
T10  -  T6  >= 5  ...... Y17

 - Nodo 11
T11  -  T10  >= 8 ...... Y18

 - Nodo final
Tfim  -  T3  >= 2 ...... Y19
Tfim  -  T5  >= 4 ...... Y20
Tfim  -  T9  >= 2 ...... Y21

\end{verbatim}

A função objetivo com os custos associados, obtidos do valor das das durações
das atividades figura da seguinte forma:

\begin{verbatim}

max : 0Y1  +  0Y2  +  4Y3  +  6Y4    
      4Y5  +  9Y6  +  4Y7  +  6Y8  +  9Y9 
      6Y10 +  8Y11 +  0Y12 +  
      5Y13 +  6Y14 +  7Y15 +  8Y16 +  
      5Y17 +  6Y18 +  2Y19 +  4Y20 + 2Y21 

\end{verbatim}

As restrições resultantes no processo anteriormente descrito são as seguintes:


\begin{verbatim}

Y1 - Y2 - Y12  <= -1 .............. Tini 

Y2 - Y3 - Y7   <= 0  .............. T0

Y3 - Y4 <= 0  ..................... T1

Y5 + Y6 - Y19  <= 0  .............. T3

- Y6 + Y7 - Y19 + Y8  <= 0 ........ T4


- Y5 + Y9 + Y1O + Y11 - Y20 <= 0 .. T5

Y12 - Y17 - Y13  <= 0 ............. T6

- Y8 - Y10 - Y14 + Y13 <= 0 ....... T7

Y14 + Y15 + Y16 - Y21 <= 0 ........ T9

- Y11 - Y16 - Y18 + Y17 <= 0 ...... T10

Y18 - Y15  <= 0 ................... T11

Y19 + Y20 + Y21 <= 1 .............. Tfim

\end{verbatim}

De igual modo, é necessário acrescer as restrições de não-negatividade. Assim:

\begin{equation*}	
	Y_i \ge 0~\forall i \in {\left[ 1, 21 \right]\cap\mathbb{N}}
\end{equation*}


\section{Conclusão}

Pode-se concluir que o modelo dual é um modelo do caminho crítico. Perante
o modelo dual construído, podem-se identificar o cumprimento dos requisitos de
um modelo do género:

\begin{itemize}
	\item Entra na rede uma unidade de fluxo.
	\item Sai da rede uma unidade de fluxo.
	\item O fluxo de entrada em cada nó é igual ao fluxo de cada nó de saída.
	\item As variáveis de decisão do modelo dual são binárias, e correspondem
		a arcos na rede, onde podem assumir o valor 1 se pertencem ao caminho,
		0 caso contrário.
	\item As durações são todas positivas na função objetivo, que correspondem
		a durações no modelo do caminho crítico.
\end{itemize}


